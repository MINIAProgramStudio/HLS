\documentclass{article}
\usepackage[utf8]{inputenc}
\usepackage[ukrainian]{babel}
\PassOptionsToPackage{hyphens}{url}\usepackage{hyperref}
\title{Проєктування високонавантаженних систем. Лабораторна 3, звіт}
\author{Михайло Голуб}
\usepackage{graphicx}
\usepackage{listings}
\usepackage{xcolor}

\definecolor{codegreen}{rgb}{0,0.6,0}
\definecolor{codegray}{rgb}{0.5,0.5,0.5}
\definecolor{codepurple}{rgb}{0.58,0,0.82}
\definecolor{backcolour}{rgb}{0.95,0.95,0.92}

\lstdefinestyle{mystyle}{
    backgroundcolor=\color{backcolour},   
    commentstyle=\color{codegreen},
    keywordstyle=\color{magenta},
    numberstyle=\tiny\color{codegray},
    stringstyle=\color{codepurple},
    basicstyle=\ttfamily\footnotesize,
    breakatwhitespace=false,         
    breaklines=true,                 
    captionpos=t,                    
    keepspaces=true,                 
    numbers=left,                    
    numbersep=5pt,                  
    showspaces=false,                
    showstringspaces=false,
    showtabs=false,                  
    tabsize=2
}
\lstset{style=mystyle}

\lstdefinelanguage{YAML}{
  keywords={true,false,null,y,n},
  keywordstyle=\color{blue}\bfseries,
  basicstyle=\ttfamily,
  sensitive=false,
  comment=[l]{\#},
  commentstyle=\color{gray}\ttfamily,
  stringstyle=\color{red}\ttfamily,
  moredelim=[l]{:},
  morestring=[b]',
  morestring=[b]"
}
\graphicspath{ {./img/} }

\def\code#1{\texttt{#1}}
\begin{document}
\maketitle
\section{Завдання лабораторної роботи}
\subsection{Загальне}
Необхідно декількома способами резалізувати оновлення значення каунтера в
СКБД PostgreSQL та оцінити час кожного із варіантів.

Таблиця \code{user_counter} з колонками \code{USER_ID}, \code{Counter}, \code{Version}.

\begin{enumerate}
    \item Lost-update
    \item Serializable update
    \item In-place update
    \item Row-level locking
    \item Optimistic concurrency control
\end{enumerate}

\subsection{Вимоги до звіту та реалізації}
\begin{itemize}
    \item Мова реалізації будь-яка
    \item Не використовувати ORM-фреймворки (Hibernate, SQLAlchemy, …)
    \item Не забувати про необхідність окремої транзакції на кожен запис
\end{itemize}

\section{Хід роботи}
\subsection{Реалізація клієнта}
\lstinputlisting[
    language=python,
    caption={lost_update.py},
    label={lst:client}
]{lost_update.py}

Приклад логу під час роботи клієнта:
\begin{verbatim}
Lost-upddate
Initialising DB
DB initialised.
Counter state: 0
Starting 10 clients x 10000 calls each...
MAIN: Threads created
0:   0%| 42/10000 [00:03<10:55, 15.20it/s]
1:   0%| 44/10000 [00:03<11:07, 14.92it/s]
2:   0%| 44/10000 [00:03<11:21, 14.62it/s]
3:   0%| 44/10000 [00:03<10:48, 15.35it/s]
4:   0%| 40/10000 [00:02<11:04, 14.98it/s]
5:   0%| 43/10000 [00:03<14:04, 11.79it/s]
6:   0%| 44/10000 [00:03<11:08, 14.90it/s]
7:   0%| 42/10000 [00:03<12:22, 13.41it/s]
8:   0%| 43/10000 [00:03<12:32, 13.22it/s]
9:   0%| 46/10000 [00:03<11:12, 14.79it/s]
\end{verbatim}

Приклад логу після завершення роботи клієнта:
\begin{verbatim}
Starting 10 clients x 10000 calls each...
MAIN: Threads created
MAIN: Threads joined
47221
Final count: 47221
Total time:  1105.88 s
Throughput:  90.43 requests/sec
\end{verbatim}


\subsection{Лічильник без блокування}
Застосунок отримує значення лічильника, інкрементує його всередині потоку застосунку
 і записує назад в Hazelcast.
Таким чином, блокування відсутні.
\lstinputlisting[
    language=python,
    caption={app\_no\_block.py},
    label={lst:appnoblock}
]{app_no_block.py}

Результати тестування:\\
Час роботи: 1106 секунд\\
Пропускна здатність: 90.43 запитів/с\\
Кінцеве значення лічильника: 47221

\subsection{Лічильник з песимістичним блокуванням}
Застосунок блокує блок (або очікує доки може заблокувати), отримує значення лічильника, 
інкрементує його всередині потоку застосунку, записує назад в Hazelcast та відпускає блок.
\lstinputlisting[
    language=python,
    caption={app\_pessimistic\_block.py},
    label={lst:apppessimisticblock}
]{app_pessimistic_block.py}
Результати тестування:\\
Час роботи: 3732 секунд\\
Пропускна здатність: 26.79 запитів/с\\
Кінцеве значення лічильника: 100к

\subsection{Лічильник з оптимістичним блокуванням}
\lstinputlisting[
    language=python,
    caption={app\_optimistic\_block.py},
    label={lst:appoptimisticblock}
]{app_optimistic_block.py}
Результати тестування:\\
Час роботи: 1964 секунди\\
Пропускна здатність: 50.91 запитів/с\\
Кінцеве значення лічильника: 100к

\subsection{CP Subsystem та IAtomicLong}
Додано файл \code{hazelcast.yaml} який вказує що необхідно активувати CP Subsystem
\lstinputlisting[
    language=yaml,
    caption={hazelcast.yaml},
    label={lst:hazelcast}
]{hazelcast.yaml}

В лозі нод наявне повідомлення:
\begin{verbatim}
CP Subsystem is enabled with 3 members.
\end{verbatim}

\lstinputlisting[
    language=python,
    caption={app\_CP.py},
    label={lst:appCP}
]{app_CP.py}

Результати тестування:\\
Час роботи: 1126 секунд\\
Пропускна здатність: 88.74 запитів/с\\
Кінцеве значення лічильника: 100к

\section{Результати}
\begin{center}
\begin{tabular}{| c | c | c | c |}
    \hline
    Варіант блокування& Час роботи & Пропускна здатність & Значення лічильника \\
    \hline
    Відсутнє & 1106 & 90.43 & 47721 \\
    \hline
    Песимістичне & 3732 & 26.79 & 100k \\
    \hline
    Оптимістичне & 1964 & 50.91 & 100k \\
    \hline
    CP Subsystem & & &\\
    IAtomicLong & 1126 & 88.74 & 100k\\
    \hline
\end{tabular}
\end{center}

\end{document}