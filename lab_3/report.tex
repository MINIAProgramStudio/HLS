\documentclass{article}
\usepackage[utf8]{inputenc}
\usepackage[ukrainian]{babel}
\PassOptionsToPackage{hyphens}{url}\usepackage{hyperref}
\title{Проєктування високонавантаженних систем. Лабораторна 3, звіт}
\author{Михайло Голуб}
\usepackage{graphicx}
\usepackage{listings}
\usepackage{xcolor}

\definecolor{codegreen}{rgb}{0,0.6,0}
\definecolor{codegray}{rgb}{0.5,0.5,0.5}
\definecolor{codepurple}{rgb}{0.58,0,0.82}
\definecolor{backcolour}{rgb}{0.95,0.95,0.92}

\lstdefinestyle{mystyle}{
    backgroundcolor=\color{backcolour},   
    commentstyle=\color{codegreen},
    keywordstyle=\color{magenta},
    numberstyle=\tiny\color{codegray},
    stringstyle=\color{codepurple},
    basicstyle=\ttfamily\footnotesize,
    breakatwhitespace=false,         
    breaklines=true,                 
    captionpos=t,                    
    keepspaces=true,                 
    numbers=left,                    
    numbersep=5pt,                  
    showspaces=false,                
    showstringspaces=false,
    showtabs=false,                  
    tabsize=2
}
\lstset{style=mystyle}

\lstdefinelanguage{YAML}{
  keywords={true,false,null,y,n},
  keywordstyle=\color{blue}\bfseries,
  basicstyle=\ttfamily,
  sensitive=false,
  comment=[l]{\#},
  commentstyle=\color{gray}\ttfamily,
  stringstyle=\color{red}\ttfamily,
  moredelim=[l]{:},
  morestring=[b]',
  morestring=[b]"
}
\graphicspath{ {./img/} }

\def\code#1{\texttt{#1}}
\begin{document}
\maketitle
\section{Завдання лабораторної роботи}
\subsection{Загальне}
Необхідно декількома способами резалізувати оновлення значення каунтера в
СКБД PostgreSQL та оцінити час кожного із варіантів.

Таблиця \code{user\_counter} з колонками \code{USER\_ID}, \code{Counter}, \code{Version}.

\begin{enumerate}
    \item Lost-update
    \item Serializable update
    \item In-place update
    \item Row-level locking
    \item Optimistic concurrency control
\end{enumerate}

\subsection{Вимоги до звіту та реалізації}
\begin{itemize}
    \item Мова реалізації будь-яка
    \item Не використовувати ORM-фреймворки (Hibernate, SQLAlchemy, \dots)
    \item Не забувати про необхідність окремої транзакції на кожен запис
\end{itemize}

\section{Хід роботи}
\subsection{Реалізація клієнта}
\lstinputlisting[
    language=python,
    caption={lost\_update\_or\_serialize.py},
    label={lst:US}
]{lost_update_or_serialize.py}

Приклад логу під час роботи клієнта:
\begin{verbatim}
Lost-upddate
Initialising DB
DB initialised.
Counter state: 0
Starting 10 clients x 10000 calls each...
MAIN: Threads created
0:   0%| 42/10000 [00:03<10:55, 15.20it/s]
1:   0%| 44/10000 [00:03<11:07, 14.92it/s]
2:   0%| 44/10000 [00:03<11:21, 14.62it/s]
3:   0%| 44/10000 [00:03<10:48, 15.35it/s]
4:   0%| 40/10000 [00:02<11:04, 14.98it/s]
5:   0%| 43/10000 [00:03<14:04, 11.79it/s]
6:   0%| 44/10000 [00:03<11:08, 14.90it/s]
7:   0%| 42/10000 [00:03<12:22, 13.41it/s]
8:   0%| 43/10000 [00:03<12:32, 13.22it/s]
9:   0%| 46/10000 [00:03<11:12, 14.79it/s]
\end{verbatim}

Приклад логу після завершення роботи клієнта:
\begin{verbatim}
Lost-upddate
Initialising DB
DB initialised.
Counter state: 0
Starting 10 clients x 10000 calls each...
MAIN: Threads created
MAIN: Threads joined
DB initialised.
Final count: 10661
Total time:  15.21 s
Throughput:  6573.22 requests/sec
\end{verbatim}

\subsection{Lost-update}
Лістинг об'єднаного lost-update та Serializable (тип роботи обирається змінною) клієнта був наведений вище.\\

Результати тестування:\\
Час роботи: 15.21 секунди\\
Пропускна здатність: 6573 запитів/с\\
Кінцеве значення лічильника: 10661

\subsection{Serializable}
При запуску таких самих запитів, але з рівнем ізоляції SERIALIZABLE,
програма видає наступну помилку на всіх воркерах, окрім одного:
\begin{verbatim}
cur.execute("UPDATE user_counter SET counter = %s WHERE user_id = %s", (counter, 1))
psycopg2.errors.SerializationFailure: 
ПОМИЛКА:  не вдалося серіалізувати доступ через паралельне оновлення
\end{verbatim}

Результати тестування:\\
Час роботи: 3 секунди\\
Пропускна здатність: --- запитів/с\\
Кінцеве значення лічильника: 10к

З певного моменту, лише один клієнт робив інкрементацію лічильника.

Таку помилкову поведінку можна виправити використавши In-place update 
або робити повторні запити у разі помилки.

\subsection{Serializable з повтором}
Якщо додати до Serializable повторний запит до таблиці у разі отримання помилки серіалізації,
можна отримати щось що схоже на оптимістичне блокування:
якщо все ок -- підтвердити операцію; 
якщо не вдалося серіалізувати операції -- повернути помилку, 
клієнти очікують випадковий час і пробують зробити операцію знову

\lstinputlisting[
    language=python,
    caption={serialize\_and\_repeat.py},
    label={lst:SR}
]{serialize_and_repeat.py}

Результати тестування:\\
Час роботи: 860 секунд\\
Пропускна здатність: 116.15 запитів/с\\
Кінцеве значення лічильника: 100к

\subsection{In-place update}
\lstinputlisting[
    language=python,
    caption={Inplace.py},
    label={lst:IPU}
]{Inplace.py}

Результати тестування:\\
Час роботи: 12.56 секунд\\
Пропускна здатність: 7963.53 запитів/с\\
Кінцеве значення лічильника: 100к

\subsection{Row-level locking}
\lstinputlisting[
    language=python,
    caption={Rowlevel.py},
    label={lst:RL}
]{Rowlevel.py}

Результати тестування:\\
Час роботи: 20.06 секунд\\
Пропускна здатність: 4985.25 запитів/с\\
Кінцеве значення лічильника: 100к

\subsection{Optimistic concurrency control}

\lstinputlisting[
    language=python,
    caption={Optimistic.py},
    label={lst:RL}
]{Optimistic.py}

Результати тестування:\\
Час роботи: 88 секунд\\
Пропускна здатність: 1136.31 запитів/с\\
Кінцеве значення лічильника: 100к

\newpage
\section{Результати}
\begin{center}
\begin{tabular}{| c | c | c | c |}
    \hline
    Варіант & Час роботи & Пропускна здатність & Значення лічильника \\
    \hline
    Lost-update & 15.21 & 6573 & 10661 \\
    \hline
    Serializable & 3 & --- & 10к \\
    \hline
    Ser. з повтором & 860 & 116.15 & 100k \\
    \hline
    Inplace update & 12.56 & 7963 & 100k\\
    \hline
    Row-level locking & 20.06 & 4985 & 100k\\
    \hline
    Optimistic concurrency control & 88 & 1136.31 & 100k\\
    \hline
    
\end{tabular}
\end{center}

\end{document}