\documentclass{article}
\usepackage[utf8]{inputenc}
\usepackage[ukrainian]{babel}
\PassOptionsToPackage{hyphens}{url}\usepackage{hyperref}
\title{Проєктування високонавантаженних систем. Лабораторна 2, звіт}
\author{Михайло Голуб}
\usepackage{graphicx}
\usepackage{listings}
\usepackage{xcolor}

\definecolor{codegreen}{rgb}{0,0.6,0}
\definecolor{codegray}{rgb}{0.5,0.5,0.5}
\definecolor{codepurple}{rgb}{0.58,0,0.82}
\definecolor{backcolour}{rgb}{0.95,0.95,0.92}

\lstdefinestyle{mystyle}{
    backgroundcolor=\color{backcolour},   
    commentstyle=\color{codegreen},
    keywordstyle=\color{magenta},
    numberstyle=\tiny\color{codegray},
    stringstyle=\color{codepurple},
    basicstyle=\ttfamily\footnotesize,
    breakatwhitespace=false,         
    breaklines=true,                 
    captionpos=t,                    
    keepspaces=true,                 
    numbers=left,                    
    numbersep=5pt,                  
    showspaces=false,                
    showstringspaces=false,
    showtabs=false,                  
    tabsize=2
}
\lstset{style=mystyle}

\lstdefinelanguage{YAML}{
  keywords={true,false,null,y,n},
  keywordstyle=\color{blue}\bfseries,
  basicstyle=\ttfamily,
  sensitive=false,
  comment=[l]{\#},
  commentstyle=\color{gray}\ttfamily,
  stringstyle=\color{red}\ttfamily,
  moredelim=[l]{:},
  morestring=[b]',
  morestring=[b]"
}
\graphicspath{ {./img/} }

\def\code#1{\texttt{#1}}
\begin{document}
\maketitle
\section{Завдання лабораторної роботи}
\subsection{Загальне}
\begin{enumerate}
    \item Встановити і налаштувати Hazelcast 5.4.x (у новіших версіях частина необхідного для виконання завдань функціоналу є платною)
    \item Сконфігурувати і запустити 3 ноди (інстанси) об'єднані в кластер або як частину Java-застосування, або як окремі застосування У справжній системі кожна нода має запускатись на окремому сервері.
    \item Далі, на основі прикладу з Distributed Map, напишіть код який буде емулювати інкремент значення для одного й того самого ключа у циклі до 10К. 
        Це необхідно робити у 10 потоках.
    \item Виходячи з того, що 10 потоків інкрементують каунтер 10К разів кожен,
        то остаточне значення каунтера має бути 10*10\_000 = 100\_000.
        Для імплементації спочатку скористаємось Distributed Map у Hazelcast
    \item Реалізуйте каунтер без блокувань. 
        Поміряйте час виконання, та подивиться чи коректне кінцеве значення каунтера ви отримаєте.
    \item Реалізуйте каунтер з використанням песимістичного блокування. 
        Поміряйте час виконання, та подивиться чи коректне кінцеве значення каунтера ви отримаєте.
    \item реалізуйте каунтер з використанням оптимістичного блокування. 
        Поміряйте час виконання, та подивиться чи коректне кінцеве значення каунтера ви отримаєте.
    \item Реалізуйте каунтер з використанням IAtomicLong та увімкнувши підтимку CP
        Sysbsystem на основі трьох нод.
        УВАГА! Без CP Sysbsystem не гарантується коректність результату 
        (у протоколі мають бути логи Hazelcast з яких видно, що CP Subsystem активована та складається з 3-х нод)
        Поміряйте час виконання, та подивиться чи коректне кінцеве значення каунтера ви отримаєте.
\end{enumerate}

\subsection{Вимоги до звіту та реалізації}
\begin{itemize}
    \item Мова реалізації будь-яка
    \item Має бути надано код програми/скрипта та результати виконання
    \item Приведно лог, який видають ноди Hazelcast, де буде видно що кластер складається з 3-х нод і що активована CP Subsystem

\end{itemize}


\section{Запуск Hazelcast }
Hazelcast локально захощено через Docker. 
Створюються три контейнери контейнери Hazelcast 5.4.0: \code{hz-node1}, \code{hz-node2}, \code{hz-node3} на портах 5701, 5702 та 5703 відповідно.
Та контейнер Hazelcast management-center 5.4.0: \code{hz-mz} на порті 8080
\lstinputlisting[
    language=yaml,
    caption={docker-compose.yml},
    label={lst:dockercompose}
]{docker-compose.yml}

Після запуску усі чотири контейнери запущені успішно і існує кластер з трьома нодами. Про це свідчить ця частина логу:
\begin{verbatim}
[ WARN] [main] [c.h.c.CPSubsystem]: [hazelcast3]:5701 [dev] [5.4.0] CP Subsystem is not enabled. CP data structures will operate in UNSAFE mode! Please note that UNSAFE mode will not provide strong consistency guarantees.
Members {size:3, ver:3} [
    Member [hazelcast1]:5701 - 360c484e-0413-4f01-b810-edd4f20fb514
    Member [hazelcast2]:5701 - 912b48a7-4b31-4303-9e30-06560e463ef1
    Member [hazelcast3]:5701 - ae62cc6a-26c6-4188-b174-ce60c234888c this
]
\end{verbatim}

\section{Клієнт}
Майже ідентичний клієнту реалізованому в першій лабораторній роботі.
Додано лог проходження певної частки запитів, для розуміння чи просувається виконання запитів.
\lstinputlisting[
    language=python,
    caption={client.py},
    label={lst:client}
]{client.py}

Приклад логу під час роботи клієнта:
\begin{verbatim}
Starting 10 clients x 10000 calls each...
MAIN: Threads created
0:   7%| 709/10000 [01:17<05:28, 28.32it/s]
1:   7%| 673/10000 [01:17<06:52, 22.62it/s]
2:   7%| 677/10000 [01:17<06:45, 22.98it/s]
3:   7%| 666/10000 [01:17<06:13, 25.02it/s]
4:   7%| 675/10000 [01:17<06:10, 25.14it/s]
5:   7%| 675/10000 [01:17<06:03, 25.67it/s]
6:   7%| 681/10000 [01:17<05:54, 26.27it/s]
7:   7%| 673/10000 [01:17<07:18, 21.25it/s]
8:   7%| 675/10000 [01:17<06:47, 22.89it/s]
9:   7%| 673/10000 [01:17<06:26, 24.10it/s]
\end{verbatim}

Приклад логу після завершення роботи клієнта:
\begin{verbatim}
Starting 10 clients x 10000 calls each...
MAIN: Threads created
MAIN: Threads joined
50663
Final count: 50663
Total time:  1184.66 s
Throughput:  84.41 requests/sec
\end{verbatim}


\section{Лічильник без блокування}
Застосунок отримує значення лічильника, інкрементує його всередині потоку застосунку
 і записує назад в Hazelcast.
Таким чином, блокування відсутні.
\lstinputlisting[
    language=python,
    caption={app\_no\_block.py},
    label={lst:appnoblock}
]{app_no_block.py}

Результати тестування:\\
Час роботи: 1184 секунди\\
Пропускна здатність: 84.41 запитів/с\\
Кінцеве значення лічильника: 32582

\section{Лічильник з песимістичним блокуванням}
Застосунок блокує блок (або очікує доки може заблокувати), отримує значення лічильника, 
інкрементує його всередині потоку застосунку, записує назад в Hazelcast та відпускає блок.
\lstinputlisting[
    language=python,
    caption={app\_pessimistic\_block.py},
    label={lst:apppessimisticblock}
]{app_pessimistic_block.py}
Задля економії часу, зменшено кількість запитів клієнтів з 10000 до 10.\\
Результати тестування:\\
Час роботи: 1290 секунд\\
Пропускна здатність: 0.08 запитів/с\\
Кінцеве значення лічильника: 100
\end{document}