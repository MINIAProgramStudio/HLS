\documentclass{article}
\usepackage[utf8]{inputenc}
\usepackage[ukrainian]{babel}
\PassOptionsToPackage{hyphens}{url}\usepackage{hyperref}
\title{Проєктування високонавантаженних систем. Лабораторна 4, звіт}
\author{Михайло Голуб}
\usepackage{graphicx}
\usepackage{listings}
\usepackage{xcolor}

\definecolor{codegreen}{rgb}{0,0.6,0}
\definecolor{codegray}{rgb}{0.5,0.5,0.5}
\definecolor{codepurple}{rgb}{0.58,0,0.82}
\definecolor{backcolour}{rgb}{0.95,0.95,0.92}

\lstdefinestyle{mystyle}{
    backgroundcolor=\color{backcolour},   
    commentstyle=\color{codegreen},
    keywordstyle=\color{magenta},
    numberstyle=\tiny\color{codegray},
    stringstyle=\color{codepurple},
    basicstyle=\ttfamily\footnotesize,
    breakatwhitespace=false,         
    breaklines=true,                 
    captionpos=t,                    
    keepspaces=true,                 
    numbers=left,                    
    numbersep=5pt,                  
    showspaces=false,                
    showstringspaces=false,
    showtabs=false,                  
    tabsize=2
}
\lstset{style=mystyle}

\lstdefinelanguage{YAML}{
  keywords={true,false,null,y,n},
  keywordstyle=\color{blue}\bfseries,
  basicstyle=\ttfamily,
  sensitive=false,
  comment=[l]{\#},
  commentstyle=\color{gray}\ttfamily,
  stringstyle=\color{red}\ttfamily,
  moredelim=[l]{:},
  morestring=[b]',
  morestring=[b]"
}
\graphicspath{ {./img/} }

\def\code#1{\texttt{#1}}
\begin{document}
\maketitle
\section{Завдання лабораторної роботи}
\subsection{Налаштування реплікації}
\begin{enumerate}
    \item Налаштувати реплікацію в конфігурації: Primary with Two Secondary Members (P-S-S) (всі ноди можуть бути запущені як окремі процеси або у Docker контейнерах)
    \item Спробувати зробити запис з однією відключеною нодою та write concern рівнім 3 та нескінченім таймаутом. Спробувати під час таймаута включити відключену ноду 
    \item Аналогічно попередньому пункту, але задати скінченний таймаут та дочекатись його закінчення. Перевірити чи данні записались і чи доступні на читання з рівнем readConcern: “majority”
    \item Продемонструвати перевибори primary node відключивши поточний primary (Replica Set Elections)	і що після відновлення роботи старої primary на неї реплікуються нові дані, які з'явилися під час її простою
\end{enumerate}

\subsection{Аналіз продуктивності та перевірка цілісності}
Аналогічно попереднім завданням, необхідно буде створити колекцію (таблицю) з каунтером лайків. Далі з 10 окремих клієнтів одночасно запустити інкерементацію каунтеру лайків по 10\_000 на кожного клієнта з різними опціями взаємодії з MongoDB.

\begin{itemize}
  \item Вказавши у парметрах \code{findOneAndUpdate writeConcern = 1} (це буде означати, що запис іде тільки на Primary ноду і не чекає відповіді від Secondary), запустіть 10 клієнтів  з інкрементом по 10\_000 на кожному з них. Виміряйте час виконання та перевірте чи кінцеве значення буде дорівнювати очікуваному - 100К
  \item Вказавши у парметрах \code{findOneAndUpdate writeConcern = majority} (це буде означати, що Primary чекає поки значення запишется на більшість нод), запустіть 10 клієнтів  з інкрементом по 10\_000 на кожному з них. Виміряйте час виконання та перевірте чи кінцеве значення буде дорівнювати очікуваному - 100К
  \item Повторно запустить код при \code{writeConcern = 1}, але тепер під час роботи відключіть Primary ноду і подивитись що буде обрана інша Primary нода, яка продовжить обробку запитів, і чи кінцевий результат буде коректним.
  \item Повторно запустить код при \code{writeConcern = majority}, але тепер під час роботи відключіть Primary ноду і подивитись що буде обрана інша Primary нода, яка продовжить обробку запитів, і чи кінцевий результат буде коректним. 
\end{itemize}

При \code{writeConcern = 1} деякі записи можуть губитись під час раптового відключення. При \code{writeConcern = majority} має виходити очікуваний результат.

\subsection{Вимоги до звіту та реалізації}
\begin{itemize}
    \item команди та результати їх виконання у вигляді скріншотів
    \item аналіз отриманих результатів

\end{itemize}

\section{Хід роботи}
\subsection{Реалізація MongoDB}
MongoDB запущена через Docker відповідним docker-compose:
\lstinputlisting[
    language=yaml,
    caption={docker-compose.yml},
    label={lst:US}
]{docker-compose.yml}

Після запуску контейнерів, виконано команду для ініціалізації:
\begin{figure}[h]
        \centering
        \includegraphics[width=\linewidth]{images/init.png}
        \caption{Ініціалізація кластеру}
\end{figure}

\begin{figure}[h]
        \centering
        \includegraphics[height=0.9\textheight]{images/status.png}
        \caption{Лог існування Primary}
        \label{fig:primary}
\end{figure}

Як видно з логу на рисунку \ref{fig:primary}, вибори проведені і Primary існує.

\subsection{Прості записи}
Виконання простих записів з різними умовами роботи нодів MongoDB виконує наступний код:
\lstinputlisting[
    language=python,
    caption={simple\_write.py},
]{simple_write.py}

Для повністю працюючого кластера код виводить:
\begin{verbatim}
Attempting insert
Write acknowledged: 693864cfa44a957b71adf265
Attempting read
Read result: {'_id': ObjectId('693864cfa44a957b71adf265'),
 'msg': 'write concern 3 test',
 'ts': datetime.datetime(2025, 12, 9, 20, 5, 3, 473000)}
Done.
\end{verbatim}

Для одного вимкненого Secondary код вічно очікував. Після увімкнення ноди:
\begin{verbatim}
Attempting insert
Write acknowledged: 693864ebf413918fc4237cd8
Attempting read
Read result: {'_id': ObjectId('693864ebf413918fc4237cd8'),
 'msg': 'write concern 3 test',
 'ts': datetime.datetime(2025, 12, 9, 20, 5, 31, 380000)}
Done.
\end{verbatim}

Скінченний таймаут для вимкненого Secondary:
\begin{verbatim}
Attempting insert
WRITE FAILED:
<class 'pymongo.errors.WTimeoutError'> waiting for replication timed out, 
full error: {'code': 64, 'codeName': 'WriteConcernFailed', 
'errmsg': 'waiting for replication timed out', 
'errInfo': {'wtimeout': True, 'writeConcern': {'w': 3, 'wtimeout': 10000, 'provenance': 'clientSupplied'}}}
\end{verbatim}

Скінченний таймаут та увімкнення Secondary під час нього:
\begin{verbatim}
Attempting insert
Write acknowledged: 693865b51c6e7fa3b7764217
Attempting read
Read result: {'_id': ObjectId('693865b51c6e7fa3b7764217'),
 'msg': 'write concern 3 test', 
 'ts': datetime.datetime(2025, 12, 9, 20, 8, 53, 70000)}
\end{verbatim}

Продемонструвати вибори не вийшло, оскільки працюючі ноди не дозволили підключення через термінал до mongosh після вимкнення Primary ноди. 
Довгий процес дебагу не виявив помилок, які могли б викликати таку поведінку :/

\subsection{Реалізація клієнта}
Клієнт запускається пайтон-скриптом, в якому в залежності від завдання виставляється відповідний \code{writeConcern}.
\lstinputlisting[
    language=python,
    caption={worker.py},
]{worker.py}


\subsection{WC = 1}
Результати тестування:\\
Час роботи: 26.37 секунд\\
Пропускна здатність: 3792.65 запитів/с\\
Кінцеве значення лічильника: 100к

\subsection{WC = 'majority'}
Результати тестування:\\
Час роботи: 76.2 секунди\\
Пропускна здатність: 1312.37 запитів/с\\
Кінцеве значення лічильника: 100к

\subsection{WC = 1 та падіння Primary}
Результати тестування:\\
Час роботи: 23.05 секунд\\
Пропускна здатність: 4338.28 запитів/с\\
Кінцеве значення лічильника: 100к

\subsection{WC = 'majority' та падіння Primary}
Результати тестування:\\
Час роботи: 59.51 секунд\\
Пропускна здатність: 1680.42 запитів/с\\
Кінцеве значення лічильника: 100к



\section{Результати}
\begin{center}
\begin{tabular}{| c | c | c | c |}
    \hline
    Варіант & Час роботи & Пропускна здатність & Значення лічильника \\
    \hline
    1 & 26.37 & 3792.65 & 100к \\
    \hline
    majority & 76.2 & 1312.37 & 100к \\
    \hline
    1 та падіння & 23.05 & 4338.28 & 100к \\
    \hline
    majority та падіння & 59.51 & 1680.42 & 100k \\
    \hline
    
\end{tabular}
\end{center}

\end{document}